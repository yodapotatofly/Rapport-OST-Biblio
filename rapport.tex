\documentclass[a4paper, 12pt]{article}

\usepackage[utf8]{inputenc}
\usepackage[T1]{fontenc}
\usepackage[french]{babel} 
\usepackage[top=35mm, bottom=35mm, left=25mm, right=25mm]{geometry}
\usepackage{geometry}
\usepackage{graphicx}
\usepackage{multirow}  
\usepackage{subfigure}
\usepackage{verbatim}
\usepackage{url}
\usepackage{algorithmic, algorithm} 
\usepackage{amsmath,amsfonts,amssymb}
\usepackage{lmodern}
\usepackage{microtype}
\usepackage{xcolor}
\usepackage{textcomp}
\usepackage{minted}
\usepackage{framed}
\usepackage{tcolorbox}
\usepackage{etoolbox}
\BeforeBeginEnvironment{minted}{\begin{tcolorbox}[left=8mm]\begin{center}}
\AfterEndEnvironment{minted}{\end{center}\end{tcolorbox}}%
\usepackage{hyperref}
\hypersetup{
    colorlinks,
    citecolor=blue,
    filecolor=black,
    linkcolor=magenta,
    urlcolor=blue,
}
\hypersetup{
pdfpagemode={},
pdfstartview={XYZ 3000 3000 0.75}
pdfstartview={XYZ left top zoom}
}
\hypersetup{
pdftitle={Template de rapport \LaTeX},
pdfsubject={Sujet du rapport, peut être vide},
pdfauthor={Premier auteur et Deuxième auteur},
pdfkeywords = {Premier mot clé, Deuxième mot clé, etc...}
}
\setcounter{secnumdepth}{3}
\usepackage{fancyhdr}
\pagestyle{fancy}
 \lhead{\leftmark}
 \rhead{}
\usepackage{pgf, tikz}
\usetikzlibrary{arrows}


\newcommand{\makelogos}{
\begin{tikzpicture}[remember picture,overlay]
\node [shift={(3 cm,-2cm)}]  at (current page.north west){
\includegraphics[scale=.25]{images/insacvl.png}
};
\node [shift={(-3.675 cm,-2cm)}]  at (current page.north east){
\begin{minipage}{.01\textwidth}
\rotatebox{90}{\scalebox{.75}{Département}~~~~}
\end{minipage}
};
\node [shift={(-3 cm,-2cm)}]  at (current page.north east){
\begin{minipage}{.1\textwidth}
\hspace{-.5cm}
\begin{eqnarray}
&\textbf{{\color{red}S}}&\!\!\!\!\!\textnormal{écurité et}\nonumber\\
&\textbf{{\color{red}T}}&\!\!\!\!\!\textnormal{echnologies}\nonumber\\
&\textbf{{\color{red}I}}&\!\!\!\!\!\!\textnormal{nformatiques}\nonumber
\end{eqnarray}
\end{minipage}
};
\end{tikzpicture}
}
 % A priori, vous n'aurez pas besoin de modifier le contenu de ce fichier :)

\begin{document}
\pagenumbering{roman} 

%%% Remarque sur la page de garde :
% Il existe une commande beaucoup plus simple : \maketitle
% Cette commande ne permet pas directement l'insertion de logo et de données 
% additionnelles sans modifier certains fichiers de configuration
% (utilisation avancée)

\begin{titlepage}
\setlength{\headheight}{0cm}
\setlength{\headsep}{0cm}
{

%%% Insertion des logos [begin]
\makelogos
%%% Insertion des logos [end]

\vspace{4cm}

\begin{center}
\fbox{ 
\begin{minipage}[h]{.9\linewidth}
\begin{center}
{\vspace*{5mm}
\huge\textbf{Rapport de [module]}\\  %%% Titre du rapport
\vspace*{5mm}}
\end{center}
\end{minipage}
}

\vspace{15mm}

Auteur(s)\\~\\
{\large 
\bsc{Nom} Prénom 1er auteur\\
\bsc{Nom} Prénom 2ème auteur}\\
~\\
\underline{STI, 4A}\\ 

\vspace{3cm}  

\textbf{Année Universitaire 2018 - 2019\\
{\tiny version : \today}}

\vspace{2cm}  

\end{center}
  
\vfill

\begin{flushleft}
	Encadrant : \textsc{Nom} Prénom
\end{flushleft}

}
\end{titlepage}

\newpage		
\tableofcontents % Insertion de la table des matières
\addcontentsline{toc}{section}{Table des matières}

% Vous pouvez également pour des rapports plus longs (des rapports de stages par exemple) insérer une table des figures
%\listoffigures
%\addcontentsline{toc}{section}{Liste des figures} 

% Voir même une liste des algorithmes
%\listofalgorithms
%\addcontentsline{toc}{section}{Liste des algorithmes}

\clearpage 

\pagenumbering{arabic} 
\section{Contexte / Introduction}
Voici une première section. Vous pouvez faire des citations en utilisant la commande \textbackslash\texttt{cite}. Par exemple \cite{DH76}.

Les informations de publication concernant le papier cité sont à placer dans un fichier \texttt{.bib}. Dans ce template, il s'agit du fichier \texttt{bibliographie.bib}.

Ces informations peuvent être obtenues sur le web, notamment ici : 
\begin{center}
\url{https://scholar.google.fr/}
\end{center}


Pour ce faire :
\begin{enumerate}
\item chercher le nom de l'article,
\item cliquez sur les guillemets,
\item puis sur BibTeX,
\item copier l'intégralité du texte dans le fichier \texttt{bibliographie.bib},
\item modifier la clé,
\item dans votre fichier \texttt{rapport.tex}, utilisez cette clé pour citer le papier.
\end{enumerate} 

\underline{Exemple :} New directions in Cryptography, de Diffie et Hellman

\begin{enumerate}
\item La recherche : \url{https://scholar.google.fr/scholar?hl=fr&as_sdt=0%2C5&q=new+directions+in+cryptography&btnG=&oq=New+directions+in+cryptography}
\item L'entrée BibTeX : \url{https://scholar.googleusercontent.com/scholar.bib?q=info:zhumlNGssTEJ:scholar.google.com/&output=citation&scisig=AAGBfm0AAAAAXJH2ZyWHII7hxbyh6hti-poxCXmdN6Bc&scisf=4&ct=citation&cd=-1&hl=fr&scfhb=1}
\item Ici, la clé par défaut est \texttt{diffie1976new}, que je modifie en \texttt{DH76}. 
\item Pour citer ce papier, dans \texttt{rapport.tex}, j'utilise la commande \textbackslash\texttt{cite}\{\texttt{DH76}\}
\end{enumerate}



Vous pouvez aussi mettre des références en URL en pied de page\footnote{Il suffit d'utiliser la commande \textbackslash\texttt{footnote} et d'inclure votre URL à l'aide de \textbackslash\texttt{footnote} : \url{https://www.latex-project.org/} ou de \textbackslash\texttt{href} : \href{https://www.latex-project.org/}{même lien}.} (mais c'est moins propre que \textbackslash\texttt{cite}).
 

Attention, les compilations \LaTeX et BibTeX peuvent être... ``capricieuses''. Je vous recommande de suivre cet ordre :
\begin{enumerate}
\item Compilation \LaTeX{} : \texttt{pdflatex -synctex=1 --shell-escape -interaction=nonstopmode rapport.tex}
\item Compilation BibTeX : \texttt{bibtex rapport.aux}
\item Compilation \LaTeX{}
\item Compilation \LaTeX{}
\end{enumerate}
Ou plus simplement, utilisez le \texttt{Makefile} fourni.

\clearpage 
\section{Problématique}

\subsection{Une première sous-partie}
\paragraph{}
Un premier paragraphe...
\paragraph{}
Un second...

\subsection{Une seconde sous-partie}
\paragraph{Inclusion d'images/screenshot} \textcolor{blue}{$\leftarrow$ on peut donner des titres aux paragraphes :)} 

Dans ce paragraphe, on va inclure une petite image (centrée) :

\begin{figure}[h]
\centering
\includegraphics[scale=.5]{images/euler.jpg}
\caption{\label{euler}Avec une légende :)}
\end{figure}

Et plus loin on peut même faire (et simplement) référence à la Figure \ref{euler} page \pageref{euler}.

Les formules de maths sont entre \$ comme ceci $\exp^{i\pi} + 1 = 0$ ou encore entre \$\$ pour les centrer : $$\sum_{n=1}^{+\infty} \frac{1}{n^2} = \frac{\pi^2}{6}.$$

On peut également utiliser un environnement dédié :

\begin{equation}
\label{eq:zsurpz}
\mathbb{Z}/p\mathbb{Z} = \left\lbrace 0, 1, \ldots, p-1  \right\rbrace
\end{equation}

Et même aligner les équations simplement et proprement avec un autre environnement :

\begin{eqnarray}
t &=& a+b+c\\
&=& d+e \\
&=& z^{x\times y}\label{eq:ici}\\
&=& \left(\frac{\delta + \omega}{\tau}\right)
\end{eqnarray}

Et faire références à ces équations \eqref{eq:zsurpz} et \eqref{eq:ici}.

\clearpage 
\section{Apports scientifiques principaux de l’article}
\subsection{Sous-section encore...}
\subsection{Et encore...}


\clearpage 
\section{Impacts de l'article}

\clearpage 
\section{Analyse critique du travail proposé}

\clearpage 
\section*{Conclusion}
\addcontentsline{toc}{section}{Conclusion}

Ce magnifique projet nous a permis -- outre de nous familiariser avec \LaTeX{} -- de \ldots

\clearpage 
\bibliographystyle{plain}
\bibliography{bibliographie}
\addcontentsline{toc}{section}{Références}


\clearpage 
\appendix
\bigskip\noindent{\Large\bf Annexes}
\addcontentsline{toc}{section}{Annexes}
\section{Algorithme qui fait quelque chose}
\begin{minted}{c}
#include<stdio.h>

int main()
{
	printf("Hello World !\n);
	return 0;
}
\end{minted}

\clearpage 
\section{Une autre annexe}

\end{document}
