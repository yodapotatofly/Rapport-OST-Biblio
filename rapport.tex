\documentclass[a4paper, 12pt]{article}

\usepackage[utf8]{inputenc}
\usepackage[T1]{fontenc}
\usepackage[french]{babel} 
\usepackage[top=35mm, bottom=35mm, left=25mm, right=25mm]{geometry}
\usepackage{geometry}
\usepackage{graphicx}
\usepackage{multirow}  
\usepackage{subfigure}
\usepackage{verbatim}
\usepackage{url}
\usepackage{algorithmic, algorithm} 
\usepackage{amsmath,amsfonts,amssymb}
\usepackage{lmodern}
\usepackage{microtype}
\usepackage{xcolor}
\usepackage{textcomp}
\usepackage{minted}
\usepackage{framed}
\usepackage{tcolorbox}
\usepackage{etoolbox}
\BeforeBeginEnvironment{minted}{\begin{tcolorbox}[left=8mm]\begin{center}}
\AfterEndEnvironment{minted}{\end{center}\end{tcolorbox}}%
\usepackage{hyperref}
\hypersetup{
    colorlinks,
    citecolor=blue,
    filecolor=black,
    linkcolor=magenta,
    urlcolor=blue,
}
\hypersetup{
pdfpagemode={},
pdfstartview={XYZ 3000 3000 0.75}
pdfstartview={XYZ left top zoom}
}
\hypersetup{
pdftitle={Template de rapport \LaTeX},
pdfsubject={Sujet du rapport, peut être vide},
pdfauthor={Premier auteur et Deuxième auteur},
pdfkeywords = {Premier mot clé, Deuxième mot clé, etc...}
}
\setcounter{secnumdepth}{3}
\usepackage{fancyhdr}
\pagestyle{fancy}
 \lhead{\leftmark}
 \rhead{}
\usepackage{pgf, tikz}
\usetikzlibrary{arrows}


\newcommand{\makelogos}{
\begin{tikzpicture}[remember picture,overlay]
\node [shift={(3 cm,-2cm)}]  at (current page.north west){
\includegraphics[scale=.25]{images/insacvl.png}
};
\node [shift={(-3.675 cm,-2cm)}]  at (current page.north east){
\begin{minipage}{.01\textwidth}
\rotatebox{90}{\scalebox{.75}{Département}~~~~}
\end{minipage}
};
\node [shift={(-3 cm,-2cm)}]  at (current page.north east){
\begin{minipage}{.1\textwidth}
\hspace{-.5cm}
\begin{eqnarray}
&\textbf{{\color{red}S}}&\!\!\!\!\!\textnormal{écurité et}\nonumber\\
&\textbf{{\color{red}T}}&\!\!\!\!\!\textnormal{echnologies}\nonumber\\
&\textbf{{\color{red}I}}&\!\!\!\!\!\!\textnormal{nformatiques}\nonumber
\end{eqnarray}
\end{minipage}
};
\end{tikzpicture}
}
 % A priori, vous n'aurez pas besoin de modifier le contenu de ce fichier :)

\begin{document}
\pagenumbering{roman} 

%%% Remarque sur la page de garde :
% Il existe une commande beaucoup plus simple : \maketitle
% Cette commande ne permet pas directement l'insertion de logo et de données 
% additionnelles sans modifier certains fichiers de configuration
% (utilisation avancée)

\begin{titlepage}
\setlength{\headheight}{0cm}
\setlength{\headsep}{0cm}
{

%%% Insertion des logos [begin]
\makelogos
%%% Insertion des logos [end]

\vspace{4cm}

\begin{center}
\fbox{ 
\begin{minipage}[h]{.9\linewidth}
\begin{center}
{\vspace*{5mm}
\huge\textbf{Rapport d'étude bibliographique}\\  %%% Titre du rapport
\vspace*{5mm}}
\emph{Access control in the internet of Things : Big challenges and new opportunities}
\vspace{5mm}
\textbf{Module Ouverture Scientifique et Technique}
\end{center}
\end{minipage}
}

\vspace{15mm}

Auteur(s)\\~\\
{\large 
\bsc{Brogat} Raphaël\\
\bsc{Ferreira} Esteban\\
\bsc{Vansimay} Léo}\\
~\\
\underline{STI, 4A}\\ 

\vspace{3cm}  

\textbf{Année Universitaire 2020 - 2021\\
{\tiny version : \today}}

\vspace{2cm}  

\end{center}
  
\vfill

\begin{flushleft}
	Encadrant : \textsc{Clemente} Patrice
\end{flushleft}

}
\end{titlepage}

\newpage		
\tableofcontents % Insertion de la table des matières
\addcontentsline{toc}{section}{Table des matières}

% Vous pouvez également pour des rapports plus longs (des rapports de stages par exemple) insérer une table des figures
%\listoffigures
%\addcontentsline{toc}{section}{Liste des figures} 

% Voir même une liste des algorithmes
%\listofalgorithms
%\addcontentsline{toc}{section}{Liste des algorithmes}

\clearpage 

\pagenumbering{arabic} 
\section{Contexte / Introduction}

\cite{AC-IoT}



\clearpage 
\section{Problématique}

\subsection{Une première sous-partie}
\paragraph{}
Un premier paragraphe...
\paragraph{}
Un second...

\subsection{Une seconde sous-partie}

\clearpage 
\section{Apports scientifiques principaux de l’article}
\subsection{Sous-section encore...}
\subsection{Et encore...}


\clearpage 
\section{Impacts de l'article}

\clearpage 
\section{Analyse critique du travail proposé}

\clearpage 
\section*{Conclusion}
\addcontentsline{toc}{section}{Conclusion}

Ce magnifique projet nous a permis -- outre de nous familiariser avec \LaTeX{} -- de \ldots

\clearpage 
\bibliographystyle{plain}
\bibliography{bibliographie}
\addcontentsline{toc}{section}{Références}


\clearpage 
\appendix
\bigskip\noindent{\Large\bf Annexes}
\addcontentsline{toc}{section}{Annexes}
\section{Algorithme qui fait quelque chose}

\clearpage 
\section{Une autre annexe}

\end{document}
